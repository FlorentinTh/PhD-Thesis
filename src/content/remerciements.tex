\begin{ack}

L'aboutissement de cette thèse  n'aurait pas été possible sans le soutien et la présence de nombreuses personnes. Malheureusement, il m'est impossible de toutes les remercier par écrit, mais je tiens tout de même à m'adresser à celles qui ont joué un rôle important tout au long du cheminement de ce doctorat.

En tout premier lieu, je souhaite remercier le professeur Sébastien Gaboury, mon directeur de thèse, sans qui ces lignes n'auraient probablement jamais été écrites. Il a été une source constante de motivation et d'inspiration et il a su être compréhensif et profondément humain lors des périodes plus difficiles que j'ai pu rencontrer. J'admire la rigueur et le professionnalisme avec lesquels il a su diriger mes travaux et je lui souhaite le meilleur pour la direction du laboratoire et ses directions de recherche futures.

Dans un second temps, je souhaite également adresser mes sincères remerciements au professeur Sylvain Hallé, co-directeur de cette thèse, pour son extrême bienveillance, son enthousiasme et sa bonne humeur.

Ensuite, je tenais aussi à exprimer mes remerciements au professeur Abdenour Bouzouane qui m'a inspiré pendant ses cours et dont j'apprécie la sérénité ainsi que la richesse des échanges que nous avons eus.

Bien évidemment, je n'oublierai pas de remercier mes amis. Aurélien, merci à toi pour ton soutien et ta confiance. Ambre va pouvoir compter sur son super papa. Valère, Baptiste, François, merci pour nos discussions, nos débats sans filtres, c'est comme ça qu'on fait avancer les choses. Merci à vous d'avoir fait en sorte que je puisse compter sur vous en cas de besoin, vous avez été des collègues et des amis en or. Kévin, bienvenue à ta petite Lily et merci pour ta bonne humeur et les raclettes. Enfin, un grand merci global à tous ceux (et celles) que je n'ai pas mentionnés, mais que je n'ai pas oubliés.

Pour finir, un dernier remerciement, et non des moindres, à ma famille qui, malgré la distance entre nous, a su rester une source d'inspiration, de soutien, d'amour et de fierté.

\end{ack}
