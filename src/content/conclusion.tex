\chapter{Conclusion Générale}
\label{chap:conclusion}

Le projet de recherche présenté dans cette thèse propose, de manière originale, de nouveaux travaux qui permettent d'améliorer le contexte de recherche actuel de la reconnaissance d'activités au sein de maisons intelligentes. En effet, les solutions existantes décrites dans le chapitre \ref{chap:2} ont démontré certaines faiblesses quant à la fiabilité de ces habitats, la plupart du temps, induit par le manque de flexibilité des architectures centralisées qui ne sont plus adaptées aux besoins actuels. Paradoxalement, le domaine des maisons intelligentes a connu, au cours de la dernière décennie, un nombre important d'innovations notamment grâce à l'avènement de l'\acs{IoT}. Par conséquent, de nouveaux types de capteurs, les \textit{wearable devices}, ont alors vu le jour afin de traiter de manière différente certaines problématiques inhérentes à la reconnaissance d'activités. Cependant, selon la diversité technologique de ces capteurs exposée dans le chapitre \ref{chap:3}, les \textit{wearable devices} souffrent d'une intégration limitée au sein des habitats intelligents existants illustrant de nouveau, un problème majeur de flexibilité dans la conception de des architectures existantes.

Afin de répondre à cette problématique, cette thèse s'est découpée en trois étapes spécifiques. La première a présenté une nouvelle méthode de reconnaissance pour améliorer l'assistance des résidents des maisons intelligentes grâce à un \textit{wearble device}. Cette étape a été nécessaire, car elle nous a permis d'identifier les différents besoins essentiels à une meilleure intégration de ces dispositifs au sein des habitats intelligents. Il est donc apparu que ceux-ci étaient davantage liés aux composants logiciels des \textit{wearble device} plus qu'au matériel. Ainsi, dans un second temps, nous nous sommes penchés sur la problématique du manque de flexibilité des architectures de ce type d'habitat en proposant une nouvelle architecture capable de proposer un meilleur niveau de fiabilité ainsi qu'une flexibilité permettant de faire cohabiter les différentes applications de la reconnaissance d'activité. Enfin, la dernière phase a permis de lier le tout en introduisant une plateforme d'apprentissage machine permettant la mise en \oe{}uvre  de processus de reconnaissance d'activités génériques c'est-à-dire, de la même façon, indépendamment du type de capteurs sur lesquels ceux-ci s'appuient.

La suite de ce chapitre va donc revenir sur la réalisation des objectifs, les limitations et les perspectives d’amélioration des travaux proposés ainsi que les apports de ce travail de recherche d’un point de vue personnel.

\section{Réalisation des objectifs}

Dans le cadre de cette thèse, l'objectif principal consistait à simplifier la mise en place de nouveaux cas d'application pour les \textit{wearable devices} au sein des habitats intelligents. Pour ce faire, il a été, dans un premier temps, nécessaire d'acquérir les connaissances en ce qui concerne la conception de ce type d'habitats. Par conséquent, le chapitre \ref{chap:2} de cette thèse a présenté les quatre grands types d'architectures de maisons intelligentes existantes en identifiant leurs principaux inconvénients. Par la suite, nous nous sommes intéressés, plus précisément, aux différents processus qui composent la reconnaissance d'activité. Enfin, le chapitre \ref{chap:3} nous a permis de mieux cibler la composition des \textit{wearable devices} en détaillant les différents capteurs que ceux-ci peuvent embarquer, mais également les communications sans fil et des méthodes d'échange de données qui sont utilisées dans leur mise en place actuelle au sein des habitats intelligents.

Dans un second temps, nous avons identifié que les \textit{wearable devices} ont souvent été utilisés dans de nombreux domaines tels que la reconnaissance de gestes et d’activités, la réhabilitation ainsi que pour la surveillance de la santé au sein des habitats intelligents \citep{Khan2016,Davis2016,Chapron2018}. Cependant, certaines pistes permettant d’améliorer l’assistance des résidents d’habitats intelligents semblaient pourtant encore inexplorées. Par conséquent, nous avons proposé, dans le chapitre \ref{chap:4}, un nouveau \textit{wearable device} qui permet de reconnaître les différents types de sols. La méthode proposée a donc fait intervenir plusieurs algorithmes et technologies qui ont été présentés dans les chapitres précédents. Grâce à cette contribution, nous avons montré qu'une telle méthode était réalisable et qu'en plus d'offrir un excellent niveau de précision, celle-ci s'inscrit parfaitement dans l'optique d'améliorer l'assistance apportée au résidents de maisons intelligentes. En effet, dans ces habitats, leurs occupants peuvent faire face à différents types de sols. Par conséquent, dans le cas de personnes en perte d'autonomie, ces sols peuvent alors représenter des dangers ou causer de la peur chez les résidents. Par exemple, il est possible de mentionner les tapis ou le carrelage mouillé dans la salle de bain. Grâce à un tel cas pratique, nous avons pu déterminer de façon plus précise les problèmes de flexibilité vis-à-vis de l'intégration des \textit{wearable devices} dans les habitats intelligents et plus spécifiquement, dans l'architecture industrielle mise en \oe{}uvre au \ac{LIARA}.

Ainsi, pour répondre à cette problématique d'intégration, la contribution présentée au chapitre \ref{chap:5} a introduit un nouveau type d'architecture d'habitat intelligent. Cette architecture, inspirée des infrastructures qu'il est possible de retrouver auprès des fournisseurs de services \textit{cloud}, demeure un système distribué qui repose sur l'utilisation de microservices. Par conséquent,  elle s'inscrit directement dans la continuité du travail proposé par \cite{Plantevin2018} qui constitue les prémices des architectures distribuées fiables et évolutives dans le contexte des habitats intelligents. Cette architecture a permis de montrer à travers différentes expérimentations que sa conception permettait à la fois d'offrir un niveau de fiabilité adéquat pour la sécurité des résidents de maisons intelligentes, ainsi qu'un niveau de flexibilité et d'évolutivité permettant de simplifier le déploiement de nouveaux composants applicatifs et d'améliorer l'interopérabilité de leur exécution.

Enfin, la dernière contribution de cette thèse et qui est présenté dans le chapitre \ref{chap:6} concerne la mise en \oe{}uvre d'un \textit{workbench} d'apprentissage machine modulaire. Bien que ce dernier ait été conçu pour être utilisé de pair avec l'architecture distribuée proposée dans le chapitre \ref{chap:5}, celui demeure exploitable sur n'importe quel type d'architecture sans contraintes particulièrement complexes de déploiement. Cet outil permet de faire le lien entre l'architecture sur laquelle il est déployé et les nouvelles applications des \textit{wearable devices} au sein des habitats intelligents. En effet, puisque ce \textit{workbench} s'appuie sur l'utilisation de modules dont l'exécution est conteneurisée dans des microservices, il permet aux chercheurs de réutiliser les composants logiciels disponibles pour réaliser les différents processus d'apprentissage pour la reconnaissance d'activités. Ainsi, il est possible de déployer aussi bien des applications nécessaires aux méthodes de reconnaissances qui utilisent \textit{wearable devices} que des capteurs statiques. De plus, l'aspect modulaire de l'outil proposé permet aux expérimentateurs d'utiliser les langages de programmation et des environnements d'exécutions avec lesquels ils sont les plus familiers ou qui demeurent les plus appropriés pour chaque application, et ce, sans impact sur le fonctionnement final du \textit{workbench}.

\section{Limitations et perspectives d'amélioration}

Bien que chacune des contributions proposées dans cette thèse apporte des réponses aux problématiques qui ont été identifiées dans le premier chapitre, il subsiste certaines limitations qui suggèrent, par conséquent, de possibles voies d'améliorations.

Dans un premier temps, en ce qui concerne le \textit{wearable device} introduit pour permettre la reconnaissance des types de sols, nous pensons que malgré la qualité des résultats obtenus, cette méthode pourrait être évaluée de façon plus approfondie en faisant intervenir un plus grand nombre de personnes. En effet, la principale limitation de cette méthode concerne le faible nombre de participants qui ont été impliqué dans les expérimentation.

Par ailleurs, nous pensons également que l'architecture distribuée mériterait d'être déployée de manière plus professionnelle. Pour ce faire, du matériel plus adapté et plus fiable que les Raspberry Pi devra être utilisé. Celui-ci pourra inclure de vrais serveurs plus puissants, des systèmes de stockage \acs{RAID} éprouvés, une installation réseau redondante dédiée et des alimentations sans interruption. Cette adaptation aura donc pour objectif principal d'offrir un environnement adéquat pour le déploiement des futurs application qui seront développées au \acs{LIARA}.

Enfin, en ce qui concerne le \textit{workbench}, nous projetons, dans un futur proche, de continuer son développement. En effet, bien que les principales fonctionnalités nécessaires à son fonctionnement soient terminées, un nombre limité de modules a cependant été proposé. Toutefois, afin que celui-ci puisse répondre aux plus grands nombres de problématiques faisant intervenir un processus d'apprentissage machine, il est envisagé de proposer plusieurs nouveaux modules dans un avenir proche. Ceux-ci devront permettre le support de différents types de données et l'intégration de processus additionnels tels que des techniques de prétraitement et de fusion des données. Ces nouveaux modules pourraient aussi intégrer de nouvelles fonctionnalités relatives aux technologies de communications sans fil, tel que le support du protocole WebSocket qui n'est pas encore abouti. De ce fait, l'intégration des \textit{wearables devices} au sein des habitats intelligents s'en retrouverait encore améliorée. En outre, une autre considération sera portée sur la conception du \textit{pipeline} d'apprentissage profond qui n'a pas encore été étudiée. Finalement, cet outil devra de la même manière que pour l'architecture être déployé dans notre laboratoire de sorte qu'il puisse être utilisé régulièrement par les différents étudiants et chercheurs.

\section{Apports personnels}

L'achèvement de ce doctorat constitue le fruit d'un long travail où rigueur, créativité, autonomie, patience et détermination ont été les maîtres mots pour y parvenir. De ce fait, ce projet m'a permis de renforcer mon expertise dans divers domaines tels que l’intelligence artificielle, la conception de dispositifs électroniques ou encore les architectures distribuées. De plus, grâce aux travaux qui ont été mis en \oe{}uvre tout au long de mon cheminement doctoral, j'ai pu perfectionner mes compétences en anglais et développer une meilleure culture scientifique de manière générale.

Pour finir, la réalisation de ce projet de recherche me motive, aujourd'hui, à poursuivre dans cette voie universitaire aussi bien pour l'aspect recherche en elle-même que pour ce qui concerne l'enseignement.

Mes derniers mots vont aux personnes qui ont été présentes tout au long de ces quatre longues années de doctorat. Je remercie encore une fois les professeurs Sébastien Gaboury et Sylvain Hallé, les personnes du département d’informatique et de mathématique de l'Université du Québec, mes collègues du \acs{LIARA}. J'en profite également pour remercier le département informatique de l'Institut Universitaire de Technologie de La Rochelle sans qui je n'aurais peut-être jamais réalisé un tel parcours.
