\begin{abstract}

Le vieillissement que connaît la population, majoritairement, celle des pays développés, entraîne inévitablement un plus grand nombre de personnes dont l'autonomie se retrouve fortement diminuée, voire complètement perdue. Ce déclin d'autonomie peut être corrélé aux maladies neurodégénératives comme la maladie d'Alzheimer, mais également à différents handicaps. Selon la gravité de la perte d'autonomie engendrée par ces pathologies, une assistance rigoureuse demeure nécessaire, car les personnes touchées requièrent une prise en charge relativement constante. Actuellement, les acteurs de la prise en charge des malades sont majoritairement leurs proches. Néanmoins, ceux-ci doivent alors assumer les conséquences aussi bien sur le plan personnel et émotionnel qu'au niveau social et financier.

Les récents progrès en matière de technologies de l'information et de microélectronique ont permis de faire émerger de nouveaux concepts comme celui de l'\ac{IAm}. Ce concept peut être traduit par une couche d'abstraction où l'informatique se retrouve au service de la communication entre les objets et les personnes. En pratique, et dans une optique de court terme, l'\acl{IAm} est appliquée dans les habitats intelligents qui demeurent actuellement d'excellents vecteurs d'assistance pour les personnes touchées par une perte d'autonomie partielle ou totale. De plus, s'ils permettent de compenser les lourdes dépenses que représente leur prise en charge pour les systèmes de santé, ils peuvent également soulager les proches aidants vis-à-vis de la quantité de stress qu'ils éprouvent. Cependant, étant donné la diversité des conceptions qui ont été proposées pour ces habitats, il peut s'avérer parfois complexe d'y intégrer de nouvelles méthodes de reconnaissance qui exploitent les technologies les plus récentes comme les \textit{wearable devices}. En outre, il est apparu que certains de ces habitats souffrent de problèmes de fiabilité qui pourraient mener à des situations dangereuses pour la sécurité des résidents.

Le projet de recherche présenté dans le cadre de cette thèse propose de nouvelles approches pour améliorer l'\acl{IAm} au sein de maisons intelligentes lorsqu'elle est réalisée avec des \textit{wearable devices}. Afin d'atteindre cet objectif, la première solution proposée introduit une nouvelle méthode de reconnaissance basée sur l'exploitation des données inertielles générées par un \textit{wearable device}. Dans une seconde phase, deux nouveaux systèmes permettant une meilleure intégration de ces nouvelles méthodes dans les habitats intelligents ont été proposés. Le premier concerne une architecture distribuée et moderne qui apporte fiabilité, sécurité et qui demeure facilement évolutive comparativement aux différentes implémentations existantes pour ces environnements. Enfin, le second système est un \textit{workbench} d'apprentissage machine générique et modulaire que nous avons appelé \ac{LE2ML}. Ainsi, cet outil représente le liant entre les nouvelles applications des \textit{wearable device} et leur utilisation au sein des habitats intelligents.

\end{abstract}
